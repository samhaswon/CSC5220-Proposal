\documentclass[letterpaper]{article}
\usepackage[utf8]{inputenc}
\usepackage{url}
\usepackage{aaai}
\usepackage{times}
\usepackage{helvet}
\usepackage{courier}
\usepackage{amsmath}
\usepackage{amssymb}

\begin{filecontents}{references_lit.bib}
@article{wang2020fuelnet,
    title={FuelNet: A precise fuel consumption prediction model using long short-term memory deep network for eco-driving},
    author={Wang, Guanqun and Zhang, Licheng and Xu, Zhigang and Hina, Syeda Mahwish and Sun, Pengpeng and Min, Haigen and Wei, Tao and Qu, Xiaobo},
    year={2020}
}

@article{abukhalil2020fuel,
  title={Fuel Consumption Using OBD-II and Support Vector Machine Model},
  author={Abukhalil, Tamer and AlMahafzah, Harbi and Alksasbeh, Malek and Alqaralleh, Bassam AY},
  journal={Journal of Robotics},
  volume={2020},
  number={1},
  pages={9450178},
  year={2020},
  publisher={Wiley Online Library}
}

@article{yen_combining_2021,
    author = {Yen, Meng-Hua and Tian, Shang-Lin and Lin, Yan-Ting and Yang, Cheng-Wei and Chen, Chi-Chun},
    title = {Combining a Universal OBD-II Module with Deep Learning to Develop an Eco-Driving Analysis System},
    journal = {Applied Sciences},
    volume = {11},
    year = {2021},
    number = {4481},
    url = {https://www.mdpi.com/2076-3417/11/10/4481},
    issn = {2076-3417}
}

@article{rykala2023modeling,
  title={Modeling vehicle fuel consumption using a low-cost OBD-II interface},
  author={Ryka{\l}a, Magdalena and Grzelak, Ma{\l}gorzata and Ryka{\l}a, {\L}ukasz and Voicu, Daniela and Stoica, Ramona-Monica},
  journal={Energies},
  volume={16},
  number={21},
  pages={7266},
  year={2023},
  publisher={MDPI}
}

@article{filla2025using,
  title={Using Weather Data for Improved Analysis of Vehicle Energy Efficiency},
  author={Filla, Reno},
  journal={Data},
  volume={10},
  number={3},
  pages={31},
  year={2025},
  publisher={MDPI}
}

@article{Manjunath2024,
    author = {Manjunath, T.K. and Ashok Kumar, P.S.},
    title = {Fuel Prediction Model for Driving Patterns Using Machine Learning Techniques},
    journal = {Journal of Computer Science},
    volume = {20},
    number = {3},
    pages = {291--302},
    year = {2024},
    doi = {10.3844/jcssp.2024.291.302},
    publisher = {Science Publications},
    url = {https://thescipub.com/pdf/jcssp.2024.291.302}
}

@article{zhang2023novel,
  title={Novel Neural-Network-Based Fuel Consumption Prediction Models Considering Vehicular Jerk},
  author={Zhang, Licheng and Ya, Jingtian and Xu, Zhigang and Easa, Said and Peng, Kun and Xing, Yuchen and Yang, Ran},
  journal={Electronics},
  volume={12},
  number={17},
  pages={3638},
  year={2023},
  publisher={MDPI}
}

@article{topic2022neural,
  title={Neural network-based prediction of vehicle fuel consumption based on driving cycle data},
  author={Topi{\'c}, Jakov and {\v{S}}kugor, Branimir and Deur, Jo{\v{s}}ko},
  journal={Sustainability},
  volume={14},
  number={2},
  pages={744},
  year={2022},
  publisher={Multidisciplinary Digital Publishing Institute}
}

@article{abediasl2024real,
  title={Real-time vehicular fuel consumption estimation using machine learning and on-board diagnostics data},
  author={Abediasl, Hamidreza and Ansari, Amir and Hosseini, Vahid and Koch, Charles Robert and Shahbakhti, Mahdi},
  journal={Proceedings of the Institution of Mechanical Engineers, Part D: Journal of Automobile Engineering},
  volume={238},
  number={12},
  pages={3779--3793},
  year={2024},
  publisher={SAGE Publications Sage UK: London, England}
}

@article{yang2022predicting,
  title={Predicting gasoline vehicle fuel consumption in energy and environmental impact based on machine learning and multidimensional big data},
  author={Yang, Yushan and Gong, Nuoya and Xie, Keying and Liu, Qingfei},
  journal={Energies},
  volume={15},
  number={5},
  pages={1602},
  year={2022},
  publisher={MDPI}
}

@article{al2007experimental,
  title={Experimental investigation of factors affecting vehicle fuel consumption},
  author={Al-Momani, Walid M and Badran, Omar},
  journal={International Journal of Mechanical and Materials Engineering},
  volume={2},
  number={2},
  pages={180--188},
  year={2007},
  publisher={University of Malaya}
}

\end{filecontents}

\frenchspacing
\setlength{\pdfpagewidth}{8.5in}
\setlength{\pdfpageheight}{11in}
\pdfinfo{
/Title (Temperature-Dependent Estimation of Vehicle Fuel Economy: Literature Review)
/Author (Sindu Chitraju, Samuel Howard, Seif Ilkbarieh, Om Solanki, Andrew Wheeler)}
\setcounter{secnumdepth}{0}  

\begin{document}

% The file aaai.sty is the style file for AAAI Press 
% proceedings, working notes, and technical reports.

% Remove the copyright from the footer
\nocopyright

\title{Temperature-Dependent Estimation of Vehicle Fuel Economy:\\Literature Review}
\author{Sindu Chitraju, Samuel Howard, Seif Ilkbarieh, Om Solanki, Andrew Wheeler\\
Tennessee Technological University
}

\maketitle

% Samuel
\section*{Introduction}

Accurate fuel consumption prediction has a variety of consequences for the daily
lives of average people and, from a broader perspective, the world. These
predictions can help individuals choose more efficient vehicles, choose more
efficient routes, drive more efficiently and safely, and save themselves money.
Applied at scale, nation states can also see a greater level of energy
independence in a world of increasing political turmoil from a reduction in the
consumption of their fleets. Finally, a reduction in consumption from
transportation directly leads to a reduction in emissions of greenhouse gases.

To these ends, there has been a great body of work towards the prediction of
fuel economy to the benefit of drivers. In this review, we explore the various
methods tried thus far, ranging from basic statistics-based modeling to more
advanced neural networks. Importantly, we identify key metrics found to most
contribute to the accurate modeling of a vehicle's fuel consumption. 

Data for this modeling comes from a variety of sources. The most important data for this
purpose comes via a vehicle's On Board Diagnostics II (OBD-II) port, standard on
gasoline-powered vehicles made after 1996 in the United States. This
specification defines a standard hardware interface for gathering key, real-time
vehicle metrics along with trouble codes stored by the car's Engine Control Unit
(ECU). This enables vehicle telematics with auxiliary hardware to collect
vehicle data for everything from vehicle tracking to driving habit feedback. 

In light of the ramifications associated with fuel consumption forecasting, many
investigations have explored a variety of approaches to formulate and enhance
vehicle efficiency prediction. The progress in data acquisition technologies,
most notably via standardized protocols such as OBD-II, has facilitated more
rigorous and adaptable analyses, thereby equipping developments in such modeling.

\section*{Importantce of Data}

% Drew
\subsection*{Experimental Investigation of Factors Affecting Vehicle Fuel Consumption}

Previous work has already been completed that evaluates a number of factors that
can influence the fuel efficiency of vehicles. The work of \cite{al2007experimental}
tests a number of parameters, including altitude, vehicle speed, and
tire pressure, to name a few. The authors evaluated a number of vehicle makes
and models, with the model years used spanning nearly a decade and a half. 

To measure fuel consumption, a flow rate sensor was integrated into the fuel line
between the fuel tank and fuel pump of the tested vehicles. Because the sensor
is part of the supplying fuel line, high-accuracy measurements can be obtained
without having to make significant modifications to more complicated components,
such as the engine. 

While this work takes into account a large number of
parameters and their impact on fuel economy, the cars utilized in the experiment
are not accurate reflections of modern internal combustion engines. This is an
important point to consider because cars manufactured before the 2000s often
utilize a mechanical carburetor for the method of fuel delivery to the engine,
while newer cars use high-precision fuel injection systems. 

Additionally, the
authors had to make modifications to the car's fuel line to integrate the sensor.
While this provides very accurate measurements, it is beyond what can be
considered an easily accessible method of measurement.

\subsection*{Predicting Gasoline Vehicle Fuel Consumption in Energy and Environmental 
Impact Based on Machine Learning and Multidimensional Big Data}

The work of \cite{yang2022predicting} evaluates the real-world fuel economy performance of vehicles 
and how it compares to the estimated values provided by car manufacturers. The authors then 
utilized a number of machine learning algorithms in an attempt to create accurate predictions 
of fuel consumption based on a number of criteria.

Real-world fuel economy data was collected via the BearOil app, containing reported fuel 
efficiency from the app’s users. Additionally, behavioral data of users’ driving habits was 
gathered using a basic twenty-question survey. Environmental data was also collected to provide 
additional parameters for the models.

The collected data was cleaned and five models were trained over it: a linear regression, Naive 
Bayes, neural network, random forest, and Light Gradient-Boosting Machine (LightGBM). 
Ten-fold cross validation was used while training all model types in order to gauge performance.

As an added step to increase the explainability of the results obtained, the weights of the 
random forest model were extracted to evaluate the influence each parameter had.
While \cite{yang2022predicting} take a very large-scale approach, having nearly one million 
records to train  models over, their scope is quite large and attempts to predict fuel 
consumption using a fair  number of qualitative data points, such as a driver’s ability. 
Additionally, the focus of their  work is on how multiple models perform in the context of 
each other, as opposed to how accurate a single model can be.

\subsection*{Experimental Investigation of the Impact of Jerk on Vehicle Fuel
Consumption Prediction}

The work of \cite{zhang2023novel} evaluates the effects of vehicle jerk, where large
changes in speed occur over a very short period of time, on fuel economy. 

The authors then attempt to use data preprocessing and machine learning models to
make accurate predictions, regardless of the presence of jerk in the data. The
authors utilized a system that integrated a global positioning system (GPS) and
on-board diagnostic (OBD) interface, allowing them to gather high-accuracy,
high-precision positional data and real-time statistics on the state of the
vehicle. 

This work more closely aligns with the proposed methodology, utilizing
a common diagnostic tool compatible with a wide variety of modern vehicles.
Where this work differs is that it utilizes multiple machine learning models and
evaluates their performance in the context of each other. Our proposal focuses
on a single model and the extent to which it can accurately predict fuel
consumption.

\section*{Utilization of On-Board Diagnostics Data}

% Samuel
\subsection*{Modeling Vehicle Fuel Consumption Using a Low-Cost OBD-II Interface}

[Rykała et al.] developed an innovative approach to modeling vehicle fuel
consumption by leveraging data collected using a low-cost OBD2 adapter alongside
onboard phone sensors. Their study aimed to explore and compare various
predictive modeling techniques to identify the most effective method, using
parameters such as GPS coordinates, driving speed, and additional signals
processed through the vehicle's diagnostic system and phone sensors. With this
data, they implemented and tested three distinct types of models: a multivariate
regression model, decision trees, and a set of 750 multilayer perceptron (MLP)
models. Among these, the MLP models demonstrated the greatest accuracy, making
them the most reliable choice for forecasting fuel consumption under diverse
driving conditions. 

They captured data from a test drive of approximately 320
kilometers (199 miles) on a selected route. Data collected includes engine speed
(RPM), vehicle speed, gear, acceleration, engine load, and road slope based on a
moving average. Of these, the most important variables were gear, speed, and
engine load. Information for the gear was determined with a linear regression of
vehicle speed and engine speed.

\subsection*{Combining a Universal OBD-II Module with Deep Learning to Develop
an Eco-Driving Analysis System}

Greater accuracy can still be gained using larger networks. [Yen et al.] explore
utilizing Elman Recurrent Neural Network (RNN) and Feed-Forward Backpropagation
(FFB) models to predict fuel consumption with the goal of progressing
energy-saving and safe-driving developments. To this end, they collected data
from several vehicles to feed these networks. Their key contribution is their
models, combined with a graphical user interface for actionable driver feedback.
This includes engine speed, vehicle speed, and other driving behavior metrics.

For their data, they used three vehicles driven on a mix of relatively flat and
mountainous roads common in western Taiwan. Each of the vehicles was driven
along the same 209-kilometer (130-mile) route. This data was cleaned and
normalized before it was used for training the models. For comparison, they
calculated the instantaneous fuel consumption using the equation: 

\[
\begin{aligned}
    \mathrm{FuelConsumption}_{\mathrm{inst}} =&~ \mathrm{RPMWeight}_{\mathrm{avg}}
    \times \mathrm{EngineLoad}_{\mathrm{avg}}\\
    &\times 2.3 \times \mathrm{FuelFactor}
    \times t(s)\times\\ 
    &\mathrm{EngineDisplacement}/7545  
\end{aligned}
\] 

\noindent
This provided them with an
accurate baseline by which to compare the various methods tested within their
paper.

% Saif

\subsection*{Fuel Consumption Using OBD-II and Support Vector Machine Model}

The authors in \cite{abukhalil2020fuel} introduce a Machine Learning (ML)-based
model to estimate real-time fuel consumption by collecting data from an OBD-II
scanner. The authors’ model combines a Support Vector Machine (SVM) algorithm
with Lagrange interpolation to analyze the relationship between engines
parameters and fuel consumption. The engine parameters include Revolutions Per
Minute (RPM) and Throttle Position Sensor (TPS). To validate their model, the
authors use Mass Air Flow (MAF)-based calculations. The authors conduct their
experiments on a fixed 66 kilometer highway route using three gasoline-powered
vehicles that have different engine displacements and horsepowers. The author’s
model was able to achieve a Root-Mean-Square Error (RMSE) score of 2.4364,
proving its potential in predicting fuel consumption. 

The data employed in this
work was collected from three test-vehicles, namely a 2017 Ford Fusion, 2016
Toyota Camry, as well as a 2006 Mercedes-Benz E280. A CDP Autocom OBD-II scanner
was connected to every car, and the data points were collected during a
40-minute drive along a steep 66 kilometer route in different areas in Jordan.
The authors were able to collect 160 sample points during the drive, and the
data points include time-series readings of engine RPM, vehicle speed, throttle
position, and corresponding fuel consumption rates. The data points were used to
build polynomial regression models and evaluate the prediction accuracy of fuel
consumption against standard manufacturer values and traditional estimation
formulas. Overall, the authors’ model was able to show the influence that engine
size has on fuel consumption in real-world scenarios.

% Om
\subsection*{Fuel Prediction Model for Driving Patterns Using Machine Learning
Techniques}

\cite{Manjunath2024} explore a machine learning-based approach to predict and improve
fuel consumption by analyzing how people drive. They used linear regression and
support vector regression to estimate fuel efficiency, with the goal of
encouraging eco-friendly driving and cutting down on greenhouse gas emissions.

To gather real-world data, they used an ELM 327 OBD-II tool connected to a
laptop, which recorded 13 different sensor readings from the car. The data was
cleaned, normalized, and fine-tuned to highlight the most relevant driving
factors before being used to train the models. 

They evaluated the models using
standard metrics like R-squared, MSE, MAE, and RMSE, and found that linear
regression performed especially well. This paper lines up closely with what
we're working on, especially since they also used an OBD-II tool for data
collection. It offers some solid ideas for how we might approach fuel efficiency
prediction. That said, they only tried two machine learning techniques---so
there may be room for us to explore and find an even better-performing model.

\section*{External Data}

\subsection*{Using Weather Data for Improved Analysis of Vehicle Energy
Efficiency}

The work of \cite{filla2025using} examines how weather conditions impact vehicle energy
efficiency, particularly for battery electric vehicles (BEVs). While previous
research has explored energy efficiency models, this paper stands out by using
national meteorological data to refine estimates of energy loss and improve
predictions of vehicle performance. 

The data sources include Swedish
institutions, as well as international ones like MET Norway and Deutscher
Wetterdienst. The study combines GNSS-based vehicle logs with weather data,
using two efficient algorithms to match weather conditions to the vehicle's
location and time. Filla compares in-vehicle sensor data to both weather station
data and interpolated weather data to assess accuracy and relevance. 

Although the research focuses on EVs and doesn't cover gasoline vehicles, it offers
valuable insight into how weather data can be used to better predict energy
efficiency. This approach could guide our project to answer the questions.

\section*{Machine Learning Applications}

% Seif
\subsection*{FuelNet: A precise fuel consumption prediction model using long
short-term memory deep network for eco-driving}

The authors in \cite{wang2020fuelnet} introduced FuelNet, which is a Long
Short-Term Memory (LSTM)-based Deep Learning (DL) model designed to predict fuel
consumption in different driving scenarios. The authors argue that there is
limited research being done on physics-based and shallow learning model. As a
result, they leverage the long-term dependency capabilities of LSTM networks to
work on their time-series vehicle data. The architecture of FuelNet is composed
of three layers and has been optimized for input efficiency, allowing it to
achieve a high Coefficient of Determination (R2) prediction score of 90.1\% while
using limited computational resources. FuelNet outperformed models like VSP,
VT-Micro, GRNN, RNN, and GRU in terms of error rates and generalizability. 

As for the data employed in this work, the authors used time-series data collected
from vehicles to train and test FuelNet. Their data specifically captures five
features, namely vehicle speed, acceleration, GPS coordinates, as well as fuel
consumption. The data they used was captured over a fixed distance of 300 meters,
and across various driving states such as high-speed, optimal-speed, and
stop-and-go conditions. Their data also supports speed ranges between 10 and 80
kilometers per hour, allowing FuelNet to learn various fuel consumption patterns.
To collect the ideal set of features for the prediction process, the authors
conducted several experiments using different input combinations. The authors
determined that speed and acceleration as input allowed FuelNet to achieve its
highest accuracy. Additionally, the data used included real-world scenarios,
such as data from Shaanxi Motor Trucks. This allowed the authors to validate the
anomaly detection capabilities of FuelNet, which helps in identifying issues
such as fuel leaks effectively. Overall, FuelNet was able to prove its
practicality in detecting abnormal fuel consumption trends, thereby enhancing
driving efficiency.

\subsection*{Real-time vehicular fuel consumption estimation using machine
learning and on-board diagnostics data}

\cite{abediasl2024real} explore machine learning for real-time fuel consumption estimation
using onboard diagnostics (OBD) data from fleet vehicles. Traditional methods,
like ECU-based estimates, can be inaccurate or expensive, so they tested four
models: Random Forest (RF), Artificial Neural Networks (ANN), Support Vector
Machines (SVM), and k-Nearest Neighbors (KNN). RF and ANN performed the best, so
they focused on comparing these two. 

They collected OBD data from different
vehicle types (sedans, SUVs, pickup trucks) with various powertrains, including
hybrids, and included cold-start conditions to capture fuel consumption during
engine warm-up. The models were trained and tested using a 70-30 data split with
five-fold cross-validation to improve reliability. 

RF handled different driving
conditions better, especially in urban areas, because it avoided overfitting and
worked well with diverse data. ANN required data normalization but still had
higher errors, particularly in stop-and-go traffic. Both models outperformed
traditional ECU-based methods, proving that machine learning is a strong option
for fuel prediction. This study is relevant to our work because it also focuses
on OBD data and machine learning for fuel estimation. Their findings suggest RF
is a strong choice, but there's room to explore additional models or
optimizations to improve accuracy further. Future research could test different
driving conditions, incorporate more OBD parameters, or use hybrid models that
combine multiple approaches for even better results.

\subsection{Neural Network-Based Prediction of Vehicle Fuel Consumption Based on
Driving Cycle Data}

\cite{topic2022neural} used neural networks to predict vehicle fuel consumption based on
driving cycle data, including speed, acceleration, and road slope. Traditional
models, like linear regression, don't always capture these factors well, so they
developed a two-stage neural network model---one part predicts driving patterns,
and the other estimates fuel use. 

They trained and tested their model using six months of GPS and CAN bus data from city buses 
in Dubrovnik. The data was broken down into smaller driving cycles, and synthetic cycles 
were generated to improve prediction accuracy. Their results showed that neural networks 
outperformed traditional models in both accuracy and speed. 

A key finding was that road slope had a big impact on fuel consumption, making it an 
important factor to include. The model also showed potential for real-world applications 
like optimizing vehicle routes, validating driving cycle data, and even estimating battery
charge levels for electric vehicles. The study suggests that using advanced machine learning 
techniques can lead to better fuel efficiency insights, helping both policymakers and 
transportation companies make informed decisions. Additionally, their approach could be 
adapted for different vehicle types or extended to analyze environmental impacts related 
to fuel consumption.

\section*{Comparisons}

% Sindu

\subsection*{OBD-II Data for Fuel Consumption Prediction}

This theme includes studies that collect vehicle data through OBD-II and predict
fuel consumption using machine learning techniques. 

[Zhang et al.] evaluates the
effects of vehicle jerk, where large changes in speed occur over a very short
period of time, on fuel economy. The authors utilized a system that integrated a
global positioning system (GPS) and on-board diagnostic (OBD) interface,
allowing them to gather high-accuracy, real-time data on vehicle status. This
work closely aligns with the proposed methodology, utilizing a common diagnostic
tool compatible with a wide variety of modern vehicles. 

[Rykała et al.]
developed an innovative approach to modeling vehicle fuel consumption by
leveraging data collected using a low-cost OBD-II adapter alongside onboard
phone sensors. They tested three distinct types of models: a multivariate
regression model, decision trees, and a set of 750 multilayer perceptron (MLP)
models. Among these, the MLP models demonstrated the greatest accuracy, making
them the most reliable choice for forecasting fuel consumption under diverse
driving conditions. 

[Yen et al.] explore utilizing Elman Recurrent Neural
Network (RNN) and Feed-Forward Backpropagation (FFB) models to predict fuel
consumption with the goal of progressing energy-saving and safe-driving
developments. They collected data from several vehicles to feed these networks.
For their data, they used three vehicles driven on a mix of relatively flat and
mountainous roads common in western Taiwan. Each of the vehicles was driven
along the same 209 kilometer (130 mile) route. 

[Manjunath and Kumar] explore a
machine learning-based approach to predict and improve fuel consumption by
analyzing how people drive. To gather real-world data, they used an ELM 327
OBD-II tool connected to a laptop, which recorded 13 different sensor readings
from the car. They used linear regression and support vector regression to
estimate fuel efficiency, with the goal of encouraging eco-friendly driving and
cutting down on greenhouse gas emissions. 

[Abukalil et al.] present a Machine
Learning (ML)-based model to estimate real-time fuel consumption by collecting
data from an OBD-II scanner. The authors' model combines a Support Vector
Machine (SVM) algorithm with Lagrange interpolation to analyze the relationship
between engine parameters and fuel consumption.The Experiments were conducted on
a fixed 66 kilometer highway route using three gasoline-powered vehicles that
have different engine displacements and horsepower. The author's model was able
to achieve a Root-Mean-Square Error (RMSE) score of 2.4364, proving its
potential in predicting fuel consumption. 

[Abediasl et al.] explore machine
learning for real-time fuel consumption estimation using onboard diagnostics
(OBD) data from fleet vehicles. They collected OBD data from different vehicle
types (sedans, SUVs, pickup trucks) with various powertrains, including hybrids,
and included cold-start conditions to capture fuel consumption during engine
warm-up. They tested four models: Random Forest (RF), Artificial Neural Networks
(ANN), Support Vector Machines (SVM), and k-Nearest Neighbors (KNN). RF and ANN
performed the best, so they focused on comparing these two. 

\subsubsection*{Comparison}~\\
All six papers leverage OBD-II diagnostics, but they differ in model, vehicle diversity,
and methods. While Abukhalil et al., Majunath and Kumar, and Abediasl et al.
favor regression and SVM-based models, Rykala et al. and Yen et al. employ deep
learning approaches like MLPs and RNNs. Zhang et al. focus on the impact of jerk,
while Abesiasl et al. look into model generalization across vehicle types and
cold-start conditions. All together, these studies confirm the value of OBD data.

% Andrew 

\subsection*{Mechanism of Prediction/Evaluation}

The papers evaluated as part of the literature review utilized a variety of
analysis and prediction techniques, with several papers using multiple methods.
The most basic technique used was simple analysis and correlation of data with
fuel consumption. This was the case in \cite{al2007experimental}, who
simply correlated how fuel consumption changes as a certain parameter, such as
speed, increases. 

\cite{filla2025using} also used analysis techniques to
examine how atmospheric and weather conditions affect the performance of a
vehicle and its fuel consumption. Weather data for the locations the vehicle was
tested in was pulled from a publicly available database and used to augment the
engine data extracted from the vehicle. 

Other approaches to analysis and prediction of fuel consumption often used machine 
learning models, with some of them comparing multiple models in order to gauge the 
performance of differing architectures. 

Linear regressions are very simple machine  learning models that attempt to establish 
a linear relationship between two or more  parameters. This was one of the model types 
evaluated by \cite{topic2022neural} when they evaluated the performance of a deep 
neural network (DNN). Linear regressions were also tested by \cite{yang2022predicting} 
as part of a series of tests using different models. 

Linear regressions were also tested in the work of
\cite{Manjunath2024}, where data was captured from the OBD and evaluated by the
model. Additionally, a support vector regression was trained and compared
against the linear regression using a number of metrics, including mean absolute
error (MAE) and mean squared error (MSE). 

Other model architectures implemented by \cite{yang2022predicting} include Naive Bayes, 
a neural network, a random forest, and LightGBM. 

A multiple regression model was used by \cite{rykala2023modeling} to attempt to establish 
more complex relationships between multiple parameters. Additionally, V-fold cross-validation 
was used to test how well the model has fit to the dataset. Other model architectures
employed by the authors of this paper include decision trees and artificial
neural networks (ANNs). 

Support vector machines (SVMs) are a basic machine learning model used by some of the authors. 
The work presented in \cite{abukhalil2020fuel} used a SVM to model and estimate fuel 
consumption based on binary data extracted from the OBD II interface. 

Long short-term memory (LSTM), a machine learning model capable of contextual analysis of data, 
was another algorithm used in some of the papers. \cite{wang2020fuelnet} used this
model architecture to predict fuel consumption at different velocities,
including at high speeds, in stop-and-go traffic, and cruising at an optimal
speed. 

Random forests and ANNs were used by \cite{abediasl2024real} to
evaluate data in real time and provide up-to-date estimations for future
consumption. A variety of data was extracted from the OBD in real time and sent
to each of the models to gauge their performance. 

Deep learning models were also used to make predictions about future fuel consumption 
based on a given set of circumstances. The authors in \cite{yen_combining_2021} used 
different methods to connect the layers of the model, and then they evaluated how the 
predicted fuel efficiency compares to the calculated value collected from sensors. The
primary model evaluated by \cite{topic2022neural} was also a deep neural
network (DNN) trained to evaluate fuel efficiency. 

The authors of \cite{zhang2023novel} tested a number of models to evaluate how vehicular 
jerk, the change in acceleration over a period of time and the mathematical derivative of 
acceleration plotted on a graph, influence the fuel efficiency of a vehicle. To do this, 
they provided a number of parameters to a LSTM, recurrent neural network (RNN), nonlinear 
auto-regressive model with exogenous inputs (NARX), and a generalized regression neural 
network (GRNN).

% Sindu

\subsection*{Datasets}

The datasets used in the reviewed papers vary significantly in terms of their
collection methods, the types of vehicles involved, and the parameters recorded.
Most studies focus on using On-Board Diagnostics (OBD) data, GPS data, or a
combination of both to predict fuel consumption. 

[Al-Momani and Badran]: This study uses real-time fuel consumption data gathered 
through a flow rate sensor integrated into the fuel line of vehicles. The dataset spans 
different vehicle makes and models, dating back to the 1990s. This approach captures fuel
consumption accurately, though it involves significant modifications to the
vehicle, making it less practical for widespread use. 

[Yang et al.]: A large-scale dataset is used, with nearly one million records gathered 
from the BearOil app. This data includes fuel efficiency reports, driving behavior data
from surveys, and environmental factors. The large dataset allows for comprehensive analysis 
and multiple machine learning models, but the focus is more on model comparison rather 
than the specific prediction of fuel consumption.

[Zhang et al.]: The data collected includes vehicle speed, acceleration, and
GPS data, which allows for real-time analysis of driving conditions, including
the effects of sudden changes in speed (jerk). Their dataset offers high
precision due to the combination of GPS and OBD data. 

[Rykała et al.]: This study utilizes a low-cost OBD-II adapter and onboard phone 
sensors to gather data from a 320 km route. Variables like engine speed, vehicle speed, 
and road slope were recorded, with the most influential parameters being gear, speed, and
engine load. 

[Yen et al.]: Three vehicles were used in their study, and data was gathered on a 
209-kilometer route through a combination of GPS and OBD data. The study uses driving 
behavior metrics, such as engine speed and vehicle speed, to predict fuel consumption 
using advanced deep learning models. 

[TK and PS]: This study collects data via an ELM 327 OBD-II tool connected to a laptop, 
which records 13 different sensor readings. Data is cleaned and normalized before being 
used to train models for fuel consumption prediction. The focus is on identifying 
eco-driving habits. 

[Filla et al.]: The authors use GPS-based vehicle logs along with national meteorological 
data to predict the impact of weather conditions on energy efficiency. While this data is 
specifically for battery electric vehicles (BEVs), it offers valuable insights into how
environmental data can complement vehicle performance data. 

[Wang et al.] (FuelNet): The FuelNet model uses time-series data, including vehicle speed,
acceleration, GPS coordinates, and fuel consumption. The dataset was gathered from real-world 
driving scenarios and spans various driving conditions, allowing for robust predictions of 
fuel consumption based on LSTM networks. 

[Abukhalil et al.]: The data used in this paper was collected using an OBD-II scanner attached
to three vehicles (Ford Fusion, Toyota Camry, and Mercedes-Benz E280). Parameters like engine 
RPM, throttle position, and fuel consumption rates were recorded during a 66-kilometer drive. 

Authors of Real-time Fuel Consumption Estimation: The dataset for this study was collected 
from fleet vehicles, capturing OBD data from various vehicle types, including hybrids, and 
tracking cold-start conditions. The data was split for training and testing four machine
learning models to estimate fuel consumption. 

Authors of Neural Network-Based Prediction: This study used six months of GPS and CAN bus data 
from city buses. The dataset includes driving cycle data such as speed, acceleration, and road
slope, which is crucial for predicting fuel consumption in real-world conditions.

\section*{Discussion}

The reviewed studies provide valuable insights into how machine learning and data collection 
methods can be used to predict vehicle fuel consumption. However, there are significant 
differences in the datasets used, which affect the robustness of the models developed. 

\subsection*{Strengths}

Most studies utilize a variety of data sources, such as OBD data, GPS data, driving behavior
data, and environmental factors. This variety helps ensure that the models account for 
different aspects of vehicle performance and driving conditions, which is essential for 
accurate predictions. Secondly, many studies leverage real-world data from fleets, consumers, 
and specific driving routes, making their models more applicable to actual conditions. 
For instance, [Yang et al.] and [Abukhalil et al.] gather large-scale, real-time data from 
various vehicles, which ensures that their models can generalize well across different vehicle 
types. Next, several studies, such as those by [Yen et al.] and [Wang et al.], explore 
advanced machine learning models (e.g., RNN and FNN), which can offer higher accuracy and
performance compared to traditional methods. These models are especially effective when 
working with large, complex datasets as we have.

\subsection*{Weaknesses}

Some studies require specialized equipment or significant  modifications to vehicles, 
which can limit the practicality of their methods. For example, [Al-Momani] and [Badran] use a 
flow rate sensor that requires integration into the vehicle’s fuel line, which is not easily 
accessible for everyday users.

While some studies focus on a broad range of parameters, others, such as TK and PS, narrow 
their focus to a small set of factors like driving habits. This may limit the models' 
ability to capture all the nuances of fuel consumption behavior. Additionally, [Filla et al.] 
focus exclusively on electric vehicles, limiting the applicability of their findings to 
gasoline-powered vehicles.

Several studies compare multiple machine learning models, such as [Yang et al.] and 
[Zhang et al.], but it is often unclear which model performs best under different 
circumstances. The use of several models can sometimes obscure the optimal choice for 
specific scenarios. [Abukhalil et al.] and Real-time Fuel Consumption Estimation studies show 
promising results with Random Forest (RF) and ANN models, but further exploration is needed 
to determine the best model for different types of vehicles and driving conditions.


\subsection*{Opportunities}

Combining multiple data sources (e.g., OBD, GPS, and weather data) could lead to more accurate 
predictions. Studies like [Filla et al.] demonstrate the potential benefits of including 
environmental data, which could be explored further in future work.

There is room to refine models by incorporating additional parameters or optimizing the 
existing models. For example, [Wang et al.] (FuelNet) use time-series data and LSTM networks, 
which are promising for improving prediction accuracy, especially in real-time applications.

Some studies, such as [Yang et al.], use massive datasets to train their models. This 
scalability could be expanded to include more vehicles, longer driving routes, and more 
diverse driving conditions to improve the robustness of the models.

The reviewed studies showcase the value of using data-driven methods to predict vehicle fuel 
consumption. However, there are clear trade-offs between model complexity, data requirements, 
and accuracy. Future work can focus on optimizing these models and making them more accessible 
for real-world applications.

% Seif, minor edits by Samuel

\section*{Conclusion}

To conclude, the literature review shows the growing importance and feasibility
of accurately predicting vehicle fuel consumption using data-driven approaches.
Many studies have proven that the integration of ML models, such as RF and SVM,
with real-world data yields high accuracy in predicting fuel efficiency across
different driving conditions. DL models, such as LSTM and MLP, have also shown
promise in predicting fuel efficiency. Moreover, sources like GPS, weather data,
as well as driving behavior surveys have shown a potential in helping capture
trends in fuel consumption. These models and data sources collectively help
drivers save money when driving by improving their vehicles' fuel efficiency,
reducing the environmental damage caused by vehicles, and helping reach
policy-related objectives. 

Although the results are promising, challenges still exist in terms of data collection 
scalability, model generalizability, and practical implementation. Studies tend to rely on 
small sample sizes, specific routes, advanced sensors, and complex modifications that limit 
their applicability. This emphasizes that further research is required to tailor an approach 
to a personalized use case with greater amounts of data. To address the limitations, we aim 
to develop a personalized system that predicts fuel economy using a more diverse set of 
features and a dataset built over approximately 7,000 miles of driving data.

\bibliographystyle{aaai} 
\bibliography{references_lit.bib}

\end{document}